% generated from JIRA project LVV
% using template at /Users/krughoff/lsst_stack/conda/miniconda3-py38_4.9.2/envs/docsteady-env/lib/python3.7/site-packages/docsteady/templates/tpr.latex.jinja2.
% using docsteady version 2.1
% Please do not edit -- update information in Jira instead
\documentclass[dm,lsstdraft,STR,toc]{lsstdoc}
\usepackage{geometry}
\usepackage{longtable,booktabs}
\usepackage{enumitem}
\usepackage{arydshln}
\usepackage{attachfile}
\usepackage{array}
\usepackage{dashrule}

\newcolumntype{L}[1]{>{\raggedright\let\newline\\\arraybackslash\hspace{0pt}}p{#1}}

\input meta.tex

\newcommand{\attachmentsUrl}{https://github.com/\gitorg/\lsstDocType-\lsstDocNum/blob/\gitref/attachments}
\providecommand{\tightlist}{
  \setlength{\itemsep}{0pt}\setlength{\parskip}{0pt}}

\setcounter{tocdepth}{4}

\begin{document}

\def\milestoneName{EFD on Summit for M1/M3}
\def\milestoneId{DM-503-EFDa}
\def\product{Data Management}

\setDocCompact{true}

\title{DM-503-EFDa: EFD on Summit for M1/M3 Test Plan and Report}
\setDocRef{\lsstDocType-\lsstDocNum}
\date{\vcsdate}
\author{ Simon Krughoff }

% Most recent last
\setDocChangeRecord{
\addtohist{}{2021-03-23}{First draft}{K. Simon Krughoff}
\addtohist{}{2021-09-21}{Draft for review}{K. Simon Krughoff}
}

\setDocCurator{K. Simon Krughoff}
\setDocUpstreamLocation{\url{https://github.com/lsst-dm/\lsstDocType-\lsstDocNum}}
\setDocUpstreamVersion{\vcsRevision}



\setDocAbstract{
This is the test plan and report for
\textbf{ EFD on Summit for M1/M3} (DM-503-EFDa),
an LSST milestone pertaining to the Data Management Subsystem.
}


\maketitle

\section{Introduction}
\label{sect:intro}


\subsection{Objectives}
\label{sect:objectives}

 The purpose of this test plan is to describe all the necessary
requirements and infrastructure for successfully testing the Engineering
Facility Database (EFD) as implemented with Kafka, InfluxDB and
Chronograf. This plan will describe the prerequisites for beginning a
test campaign, step by step instructions for each test can and a
description of the expected results and test
artifacts.\\[2\baselineskip]NB: The use of the term reliability in this
document is intended to indicate the number of messages produced
relative to the number of messages recorded in the EFD. The system shall
be considered reliable if at least XX\% of produced messages are
recorded.\\[2\baselineskip]The highest level description of this test
plan is to run the M1/M3 subsystem in an active state for no less than 5
contiguous days. During this time, all telemetry produced by the M1/M3
subsystem will appear in the InfluxDB instance running at the summit
with latency less than XX second XX\%. The maximum latency shall be less
than XX seconds. All telemetry shall also be available for interrogation
by Chronograf on similar time scales. Any gap in telemetry or
dropped/missing messages will be considered a deviation. Successful
completion of the test campaign will show that:\\

\begin{enumerate}
\tightlist
\item
  ~users are able to access data ingested in the InfluxDB at the summit
  in near real time from the Chronograf interface
\item
  the latency from message production to ingestion in InfluxDB is less
  than the nominal limit (XX sec) XX\% of the time and never more than
  XX seconds
\item
  the M1/M3 telemetry is successfully being mirrored to another influxDB
  instance at the data facility with latency less than XX hour(s)
\item
  users are able to access and analyze telemetry data from the M1/M3
  subsystem from notebooks running in the notebook aspect of the RSP
  both at the summit and at the data facility\\
\end{enumerate}



\subsection{System Overview}
\label{sect:systemoverview}

 The tests will be carried out from within an instance of the notebook
aspect of the RSP running at either the summit or the data facility. ~An
appropriate weekly version of the stack will be chosen.


\subsection{Document Overview}
\label{sect:docoverview}

This document was generated from Jira, obtaining the relevant information from the
\href{https://jira.lsstcorp.org/secure/Tests.jspa\#/testPlan/LVV-P78}{LVV-P78}
~Jira Test Plan and related Test Cycles (
\href{https://jira.lsstcorp.org/secure/Tests.jspa\#/testCycle/LVV-C163}{LVV-C163}
).

Section \ref{sect:intro} provides an overview of the test campaign, the system under test (\product{}),
the applicable documentation, and explains how this document is organized.
Section \ref{sect:testplan} provides additional information about the test plan, like for example the configuration
used for this test or related documentation.
Section \ref{sect:personnel} describes the necessary roles and lists the individuals assigned to them.

Section \ref{sect:overview} provides a summary of the test results, including an overview in Table \ref{table:summary},
an overall assessment statement and suggestions for possible improvements.
Section \ref{sect:detailedtestresults} provides detailed results for each step in each test case.

The current status of test plan \href{https://jira.lsstcorp.org/secure/Tests.jspa\#/testPlan/LVV-P78}{LVV-P78} in Jira is \textbf{ Draft }.

\subsection{References}
\label{sect:references}
\renewcommand{\refname}{}
\bibliography{lsst,refs,books,refs_ads,local}


\newpage
\section{Test Plan Details}
\label{sect:testplan}


\subsection{Data Collection}

  Observing is not required for this test campaign.

\subsection{Verification Environment}
\label{sect:hwconf}
  The environ

  \subsection{Entry Criteria}
  \begin{enumerate}
\tightlist
\item
  Before beginning this test, as set of viability tests shall be
  performed. These will show:

  \begin{enumerate}
  \tightlist
  \item
    The system demonstrates reliability (number of recorded
    messages/number of produced messages) of greater than XX\%
  \item
    The typical latency of the system is less than XX sec for a pre
    defined set of topics
  \item
    The summit data is being replicated to the instance at NCSA
  \item
    Chronograf is set up and running at both the summit and NCSA
  \end{enumerate}
\item
  The summit network and Kubernetes cluster are performing nominally
\item
  The EFD is deployed in the summit Kubernetes cluster
\item
  The M1M3 sub-component is reliably producing telemetry via Kafka
  producers with correct versions of the schema
\item
  The notebook aspect of the RSP is deployed in the summit Kubernetes
  cluster
\item
  The summit EFD is reliably replicated to an EFD instance running in a
  data facility
\item
  The notebook aspect of the RSP is deployed in the same data facility
  as that running the replicated EFD
\item
  The most recent version of the EFD client python modules are installed
  in the various deployed notebook aspects
\item
  A requirement test that the system demonstrates reliability of greater
  than XX\%
\end{enumerate}



\subsection{Related Documentation}


No additional documentation provided.


\subsection{PMCS Activity}

Primavera milestones related to the test campaign:
\begin{itemize}
\item DM-503-EFDa
\end{itemize}


\newpage
\section{Personnel}
\label{sect:personnel}

The personnel involved in the test campaign is shown in the following table.

{\small
\begin{longtable}{p{3cm}p{3cm}p{3cm}p{6cm}}
\hline
\multicolumn{2}{r}{T. Plan \href{https://jira.lsstcorp.org/secure/Tests.jspa\#/testPlan/LVV-P78}{LVV-P78} owner:} &
\multicolumn{2}{l}{\textbf{ Simon Krughoff } }\\\hline
\multicolumn{2}{r}{T. Cycle \href{https://jira.lsstcorp.org/secure/Tests.jspa\#/testCycle/LVV-C163}{LVV-C163} owner:} &
\multicolumn{2}{l}{\textbf{
Simon Krughoff }
} \\\hline
\textbf{Test Cases} & \textbf{Assigned to} & \textbf{Executed by} & \textbf{Additional Test Personnel} \\ \hline
\href{https://jira.lsstcorp.org/secure/Tests.jspa#/testCase/LVV-T2111}{LVV-T2111}
& {\small Simon Krughoff } & {\small  } &
\begin{minipage}[]{6cm}
\smallskip
{\small  }
\medskip
\end{minipage}
\\ \hline
\href{https://jira.lsstcorp.org/secure/Tests.jspa#/testCase/LVV-T2112}{LVV-T2112}
& {\small Simon Krughoff } & {\small  } &
\begin{minipage}[]{6cm}
\smallskip
{\small  }
\medskip
\end{minipage}
\\ \hline
\href{https://jira.lsstcorp.org/secure/Tests.jspa#/testCase/LVV-T2117}{LVV-T2117}
& {\small Simon Krughoff } & {\small  } &
\begin{minipage}[]{6cm}
\smallskip
{\small  }
\medskip
\end{minipage}
\\ \hline
\href{https://jira.lsstcorp.org/secure/Tests.jspa#/testCase/LVV-T2116}{LVV-T2116}
& {\small Simon Krughoff } & {\small  } &
\begin{minipage}[]{6cm}
\smallskip
{\small  }
\medskip
\end{minipage}
\\ \hline
\href{https://jira.lsstcorp.org/secure/Tests.jspa#/testCase/LVV-T2115}{LVV-T2115}
& {\small Simon Krughoff } & {\small  } &
\begin{minipage}[]{6cm}
\smallskip
{\small  }
\medskip
\end{minipage}
\\ \hline
\end{longtable}
}

\newpage

\section{Test Campaign Overview}
\label{sect:overview}

\subsection{Summary}
\label{sect:summarytable}

{\small
\begin{longtable}{p{2cm}cp{2.3cm}p{8.6cm}p{2.3cm}}
\toprule
\multicolumn{2}{r}{ T. Plan \href{https://jira.lsstcorp.org/secure/Tests.jspa\#/testPlan/LVV-P78}{LVV-P78}:} &
\multicolumn{2}{p{10.9cm}}{\textbf{ DM-503-EFDa: EFD on Summit for M1/M3 }} & Draft \\\hline
\multicolumn{2}{r}{ T. Cycle \href{https://jira.lsstcorp.org/secure/Tests.jspa\#/testCycle/LVV-C163}{LVV-C163}:} &
\multicolumn{2}{p{10.9cm}}{\textbf{ DM-503-EFDa: EFD on Summit for M1/M3 }} & Not Executed \\\hline
\textbf{Test Cases} &  \textbf{Ver.} & \textbf{Status} & \textbf{Comment} & \textbf{Issues} \\\toprule
\href{https://jira.lsstcorp.org/secure/Tests.jspa#/testCase/LVV-T2111}{LVV-T2111}
&  1
& Not Executed &
\begin{minipage}[]{9cm}
\smallskip

\medskip
\end{minipage}
&   \\\hline
\href{https://jira.lsstcorp.org/secure/Tests.jspa#/testCase/LVV-T2112}{LVV-T2112}
&  1
& Not Executed &
\begin{minipage}[]{9cm}
\smallskip

\medskip
\end{minipage}
&   \\\hline
\href{https://jira.lsstcorp.org/secure/Tests.jspa#/testCase/LVV-T2117}{LVV-T2117}
&  1
& Not Executed &
\begin{minipage}[]{9cm}
\smallskip

\medskip
\end{minipage}
&   \\\hline
\href{https://jira.lsstcorp.org/secure/Tests.jspa#/testCase/LVV-T2116}{LVV-T2116}
&  1
& Not Executed &
\begin{minipage}[]{9cm}
\smallskip

\medskip
\end{minipage}
&   \\\hline
\href{https://jira.lsstcorp.org/secure/Tests.jspa#/testCase/LVV-T2115}{LVV-T2115}
&  1
& Not Executed &
\begin{minipage}[]{9cm}
\smallskip

\medskip
\end{minipage}
&   \\\hline
\caption{Test Campaign Summary}
\label{table:summary}
\end{longtable}
}

\subsection{Overall Assessment}
\label{sect:overallassessment}

Not yet available.

\subsection{Recommended Improvements}
\label{sect:recommendations}

Not yet available.

\newpage
\section{Detailed Test Results}
\label{sect:detailedtestresults}

\subsection{Test Cycle LVV-C163 }

Open test cycle {\it \href{https://jira.lsstcorp.org/secure/Tests.jspa#/testrun/LVV-C163}{DM-503-EFDa: EFD on Summit for M1/M3}} in Jira.

Test Cycle name: DM-503-EFDa: EFD on Summit for M1/M3\\
Status: Not Executed



\subsubsection{Software Version/Baseline}
Not provided.

\subsubsection{Configuration}
Not provided.

\subsubsection{Test Cases in LVV-C163 Test Cycle}

\paragraph{ LVV-T2111 - Access to M1/M3 telemetry data in near real time via the Chronograf
interface }\mbox{}\\

Version \textbf{1}.
Open  \href{https://jira.lsstcorp.org/secure/Tests.jspa#/testCase/LVV-T2111}{\textit{ LVV-T2111 } }
test case in Jira.

Show that users can get access to visualizations in Chronograf with
telemetry arriving in less than XX seconds from a command run on the DDS
network.

\textbf{ Preconditions}:\\
See prerequisites in the Test Plan LVV-P78

Execution status: {\bf Not Executed }

Final comment:\\


Detailed steps results:

\begin{tabular}{p{2cm}p{14cm}}
\toprule
Step 1 & Step Execution Status: \textbf{ Not Executed } \\ \hline
\end{tabular}
 Description \\
{\footnotesize
Log in to whatever VPNs are necessary to both see Chronograf at the
summit and the control network necessary for commanding components fo
the M1/M3 subsystem

}
\hdashrule[0.5ex]{\textwidth}{1pt}{3mm}
  Expected Result \\
{\footnotesize
VPN connects are live

}
\hdashrule[0.5ex]{\textwidth}{1pt}{3mm}
  Actual Result \\
{\footnotesize

}
\begin{tabular}{p{2cm}p{14cm}}
\toprule
Step 2 & Step Execution Status: \textbf{ Not Executed } \\ \hline
\end{tabular}
 Description \\
{\footnotesize
Log in to chronograf running at the summit. The endpoint is currently
\href{https://chronograf-summit-efd.lsst.codes/}{https://chronograf-summit-efd.lsst.codes},
though \href{https://sqr-034.lsst.io/}{https://sqr-034.lsst.io} shall be
considered the primary source of truth for service endpoints relating to
the EFD

}
\hdashrule[0.5ex]{\textwidth}{1pt}{3mm}
  Expected Result \\
{\footnotesize
The browser showing the front page of the chronograf interface

}
\hdashrule[0.5ex]{\textwidth}{1pt}{3mm}
  Actual Result \\
{\footnotesize

}
\begin{tabular}{p{2cm}p{14cm}}
\toprule
Step 3 & Step Execution Status: \textbf{ Not Executed } \\ \hline
\end{tabular}
 Description \\
{\footnotesize
Using the ``Explore'' tab in chronograph enter a query that populates
the graph window for one value: e.g.
lsst.sal.MTM1M3.forceActuatorData.forceMagnitude. This may change
depending on what command is given to the subsystem

}
\hdashrule[0.5ex]{\textwidth}{1pt}{3mm}
  Expected Result \\
{\footnotesize
There should be a graph showing a trace of the chosen field over the
default time window (now() - 1hr)

}
\hdashrule[0.5ex]{\textwidth}{1pt}{3mm}
  Actual Result \\
{\footnotesize

}
\begin{tabular}{p{2cm}p{14cm}}
\toprule
Step 4 & Step Execution Status: \textbf{ Not Executed } \\ \hline
\end{tabular}
 Description \\
{\footnotesize
Set the refresh in chronograph to the minimum value (currently 5
seconds) and adjust the time window to show the past 15 minutes of data.
Both of these operations are done using buttons immediately above the
graph window on the right side

}
\hdashrule[0.5ex]{\textwidth}{1pt}{3mm}
  Expected Result \\
{\footnotesize
\begin{itemize}
\tightlist
\item
  The trace is now the last 15 minutes and the view refreshes every 5
  seconds
\item
  The query used to produce the trace plot will be captured as an
  artifact
\end{itemize}

}
\hdashrule[0.5ex]{\textwidth}{1pt}{3mm}
  Actual Result \\
{\footnotesize

}
\begin{tabular}{p{2cm}p{14cm}}
\toprule
Step 5 & Step Execution Status: \textbf{ Not Executed } \\ \hline
\end{tabular}
 Description \\
{\footnotesize
Issue a command to the M1/M3 subsystem to e.g. modify the force
actuators and confirm the message was successfully sent and acted upon

}
\hdashrule[0.5ex]{\textwidth}{1pt}{3mm}
  Expected Result \\
{\footnotesize
\begin{itemize}
\tightlist
\item
  Confirmation that the command was acted upon
\item
  This will also include a qualitative description of whether the
  command produced the expected change in telemetry.
\end{itemize}

}
\hdashrule[0.5ex]{\textwidth}{1pt}{3mm}
  Actual Result \\
{\footnotesize

}
\begin{tabular}{p{2cm}p{14cm}}
\toprule
Step 6 & Step Execution Status: \textbf{ Not Executed } \\ \hline
\end{tabular}
 Description \\
{\footnotesize
Verify new values are showing up in the summit instance of InfluxDB by
observing that values with newer timestamps are appearing in the
Chronograf visualization

}
\hdashrule[0.5ex]{\textwidth}{1pt}{3mm}
  Expected Result \\
{\footnotesize
Screenshots or similar documenting dashboards before and after the
command to M1/M3 is made.

}
\hdashrule[0.5ex]{\textwidth}{1pt}{3mm}
  Actual Result \\
{\footnotesize

}
\begin{tabular}{p{2cm}p{14cm}}
\toprule
Step 7 & Step Execution Status: \textbf{ Not Executed } \\ \hline
\end{tabular}
 Description \\
{\footnotesize
Produce a report for the test

}
\hdashrule[0.5ex]{\textwidth}{1pt}{3mm}
  Expected Result \\
{\footnotesize
\begin{itemize}
\tightlist
\item
  A document explaining the procedure including the topics monitored,
  the command issued and the query used to do the visualization
\end{itemize}

}
\hdashrule[0.5ex]{\textwidth}{1pt}{3mm}
  Actual Result \\
{\footnotesize

}

\paragraph{ LVV-T2112 - Latency from production to ingestion and telemetry can be analyzed via
notebooks: High and Low Cadence }\mbox{}\\

Version \textbf{1}.
Open  \href{https://jira.lsstcorp.org/secure/Tests.jspa#/testCase/LVV-T2112}{\textit{ LVV-T2112 } }
test case in Jira.

Measure the latency of production to ingestion in the EFD and show it is
less than XX seconds XX\% of the time.\\
Show that this analysis can be completed via a notebook running in an
instance of the notebook aspect of the RSP.

\textbf{ Preconditions}:\\
See prerequisites in the Test Plan LVV-P78

Execution status: {\bf Not Executed }

Final comment:\\


Detailed steps results:

\begin{tabular}{p{2cm}p{14cm}}
\toprule
Step 1 & Step Execution Status: \textbf{ Not Executed } \\ \hline
\end{tabular}
 Description \\
{\footnotesize
Log in to whatever VPNs are necessary to access to the summit notebook
aspect of the RSP

}
\hdashrule[0.5ex]{\textwidth}{1pt}{3mm}
  Expected Result \\
{\footnotesize
VPN connection is active

}
\hdashrule[0.5ex]{\textwidth}{1pt}{3mm}
  Actual Result \\
{\footnotesize

}
\begin{tabular}{p{2cm}p{14cm}}
\toprule
Step 2 & Step Execution Status: \textbf{ Not Executed } \\ \hline
\end{tabular}
 Description \\
{\footnotesize
Log in to the summit notebook aspect: https://summit-lsp.lsst.codes/nb\\
Make sure to choose a recent weekly and a large instance

}
\hdashrule[0.5ex]{\textwidth}{1pt}{3mm}
  Expected Result \\
{\footnotesize
The JupyterLab interface is displayed in the browser

}
\hdashrule[0.5ex]{\textwidth}{1pt}{3mm}
  Actual Result \\
{\footnotesize

}
\begin{tabular}{p{2cm}p{14cm}}
\toprule
Step 3 & Step Execution Status: \textbf{ Not Executed } \\ \hline
\end{tabular}
 Description \\
{\footnotesize
Open a notebook:

\begin{enumerate}
\tightlist
\item
  Navigate to the File-\textgreater{}New-\textgreater{}Notebook
\item
  When prompted, select the LSST kernel
\end{enumerate}

}
\hdashrule[0.5ex]{\textwidth}{1pt}{3mm}
  Expected Result \\
{\footnotesize
An empty notebook running in the LSST kernel

}
\hdashrule[0.5ex]{\textwidth}{1pt}{3mm}
  Actual Result \\
{\footnotesize

}
\begin{tabular}{p{2cm}p{14cm}}
\toprule
Step 4 & Step Execution Status: \textbf{ Not Executed } \\ \hline
\end{tabular}
 Description \\
{\footnotesize
Connect to the summit EFD

}
\hdashrule[0.5ex]{\textwidth}{1pt}{3mm}
  Example Code \\
{\footnotesize
from lsst\_efd\_client import EfdClient\\
efd = EfdClient('summit\_efd')

}
\hdashrule[0.5ex]{\textwidth}{1pt}{3mm}
  Expected Result \\
{\footnotesize
A notebook with an instance of the `EfdClient` configured to talk to the
summit EFD

}
\hdashrule[0.5ex]{\textwidth}{1pt}{3mm}
  Actual Result \\
{\footnotesize

}
\begin{tabular}{p{2cm}p{14cm}}
\toprule
Step 5 & Step Execution Status: \textbf{ Not Executed } \\ \hline
\end{tabular}
 Description \\
{\footnotesize
Choose a topic to query and select a 5 day window of data. The topic and
window are arbitrary, but it shall be explicit (not relative to now())
so that it can reproduced. The topic shall be one of high enough cadence
that there are many measurements for the topic, i.e. not a command or
log topic that could potentially be very sparsely populated over the 5
day window. A high cadence, ideally greater than XX Hz, topic will give
the most precise measurement of the distribution of the latency.

}
\hdashrule[0.5ex]{\textwidth}{1pt}{3mm}
  Expected Result \\
{\footnotesize
\begin{itemize}
\tightlist
\item
  A table-like object in memory containing data from the chosen topic
  and time window.
\item
  The window and topic are artifacts to be preserved
\end{itemize}

}
\hdashrule[0.5ex]{\textwidth}{1pt}{3mm}
  Actual Result \\
{\footnotesize

}
\begin{tabular}{p{2cm}p{14cm}}
\toprule
Step 6 & Step Execution Status: \textbf{ Not Executed } \\ \hline
\end{tabular}
 Description \\
{\footnotesize
\begin{enumerate}
\tightlist
\item
  The total latency is the time from the message being published,
  private\_sndStamp, and when it is ingested in the influxDB database.
  Currently the index timestamp is private\_sndStamp There will be an
  additional field added to the measurement to reflect the ingest
  timestamp. This may be technically difficult, in which case a subset
  of topics will have an extra column added by hand for the purpose of
  this analysis. Care shall also be taken to ensure the messages are in
  the same (TAI) time system. In the past some CSCs have been reporting
  TAI and some report UTC. Currently, the difference is 37 seconds.
\item
  Compute the total latency by taking the difference of the two columns,
  arr{[}`timestamp'{]} - arr{[}`private\_sndStamp'{]} (this is where
  correction for TAI/UTC would be included if necessary). The result
  shall be in seconds.
\end{enumerate}

}
\hdashrule[0.5ex]{\textwidth}{1pt}{3mm}
  Expected Result \\
{\footnotesize
An array-like object in memory containing the latency in seconds for
every message in the window

}
\hdashrule[0.5ex]{\textwidth}{1pt}{3mm}
  Actual Result \\
{\footnotesize

}
\begin{tabular}{p{2cm}p{14cm}}
\toprule
Step 7 & Step Execution Status: \textbf{ Not Executed } \\ \hline
\end{tabular}
 Description \\
{\footnotesize
Count the number of entries less than XX second(s) and divide that by
the total number of entries. This value shall be greater than or equal
to XX.

}
\hdashrule[0.5ex]{\textwidth}{1pt}{3mm}
  Expected Result \\
{\footnotesize
\begin{itemize}
\tightlist
\item
  A plot showing a histogram of the latency values indicating the XX\%
  value.
\item
  If the XX\% latency is greater than XX sec, an explanation shall be
  supplied describing why the latency is higher than expected more often
  than expected
\item
  The plot shall also indicate the maximum latency observed
\item
  If the maximum latency is above the nominal maximum (XX sec), an
  explanation shall be provided
\end{itemize}

}
\hdashrule[0.5ex]{\textwidth}{1pt}{3mm}
  Actual Result \\
{\footnotesize

}
\begin{tabular}{p{2cm}p{14cm}}
\toprule
Step 8 & Step Execution Status: \textbf{ Not Executed } \\ \hline
\end{tabular}
 Description \\
{\footnotesize
Choose a topic to query and select a 5 day window of data. The topic and
window are arbitrary, but it shall be explicit (not relative to now())
so that it can reproduced. The topic shall be low latency in order to to
measure the impact of publishing low and high latency topics
simultaneously

}
\hdashrule[0.5ex]{\textwidth}{1pt}{3mm}
  Expected Result \\
{\footnotesize
\begin{itemize}
\tightlist
\item
  A table-like object in memory containing data from the chosen topic
  and time window.
\item
  The window and topic are artifacts to be preserved
\end{itemize}

}
\hdashrule[0.5ex]{\textwidth}{1pt}{3mm}
  Actual Result \\
{\footnotesize

}
\begin{tabular}{p{2cm}p{14cm}}
\toprule
Step 9 & Step Execution Status: \textbf{ Not Executed } \\ \hline
\end{tabular}
 Description \\
{\footnotesize
\begin{enumerate}
\tightlist
\item
  The total latency is the time from the message being published,
  private\_sndStamp, and when it is ingested in the influxDB database.
  Currently the index timestamp is private\_sndStamp There will be an
  additional field added to the measurement to reflect the ingest
  timestamp. This may be technically difficult, in which case a subset
  of topics will have an extra column added by hand for the purpose of
  this analysis. Care shall also be taken to ensure the messages are in
  the same (TAI) time system. In the past some CSCs have been reporting
  TAI and some report UTC. Currently, the difference is 37 seconds.
\item
  Compute the total latency by taking the difference of the two columns,
  arr{[}`timestamp'{]} - arr{[}`private\_sndStamp'{]} (this is where
  correction for TAI/UTC would be included if necessary). The result
  shall be in seconds.
\end{enumerate}

}
\hdashrule[0.5ex]{\textwidth}{1pt}{3mm}
  Expected Result \\
{\footnotesize
An array-like object in memory containing the latency in seconds for
every message in the window

}
\hdashrule[0.5ex]{\textwidth}{1pt}{3mm}
  Actual Result \\
{\footnotesize

}
\begin{tabular}{p{2cm}p{14cm}}
\toprule
Step 10 & Step Execution Status: \textbf{ Not Executed } \\ \hline
\end{tabular}
 Description \\
{\footnotesize
Count the number of entries less than XX second(s) and divide that by
the total number of entries. This value shall be greater than or equal
to XX. ~This is intended to show that even when publishing
simultaneously with high and low cadence topics we still meet the
latency goal for both topics

}
\hdashrule[0.5ex]{\textwidth}{1pt}{3mm}
  Expected Result \\
{\footnotesize
\begin{itemize}
\tightlist
\item
  A plot showing a histogram of the latency values indicating the XX\%
  value.
\item
  If the XX\% latency is greater than XX sec, an explanation shall be
  supplied describing why the latency is higher than expected more often
  than expected
\item
  The plot shall also indicate the maximum latency observed
\item
  If the maximum latency is above the nominal maximum (XX sec), an
  explanation shall be provided
\end{itemize}

}
\hdashrule[0.5ex]{\textwidth}{1pt}{3mm}
  Actual Result \\
{\footnotesize

}
\begin{tabular}{p{2cm}p{14cm}}
\toprule
Step 11 & Step Execution Status: \textbf{ Not Executed } \\ \hline
\end{tabular}
 Description \\
{\footnotesize
Document the procedure including latency distributions, time window, and
both high and low cadence topics

}
\hdashrule[0.5ex]{\textwidth}{1pt}{3mm}
  Expected Result \\
{\footnotesize
\begin{itemize}
\tightlist
\item
  A document describing the process including the topic chosen and the
  time window.
\item
  The document shall be in the form on a notebook with saved outputs, or
  an instance of an nbreport.
\end{itemize}

}
\hdashrule[0.5ex]{\textwidth}{1pt}{3mm}
  Actual Result \\
{\footnotesize

}

\paragraph{ LVV-T2117 - Latency from production to ingestion and telemetry can be analyzed via
notebooks: Low Cadence }\mbox{}\\

Version \textbf{1}.
Open  \href{https://jira.lsstcorp.org/secure/Tests.jspa#/testCase/LVV-T2117}{\textit{ LVV-T2117 } }
test case in Jira.

Measure the latency of production to ingestion in the EFD and show it is
less than XX seconds XX\% of the time.\\
Show that this analysis can be completed via a notebook running in an
instance of the notebook aspect of the RSP.

\textbf{ Preconditions}:\\
See prerequisites in the Test Plan LVV-P78

Execution status: {\bf Not Executed }

Final comment:\\


Detailed steps results:

\begin{tabular}{p{2cm}p{14cm}}
\toprule
Step 1 & Step Execution Status: \textbf{ Not Executed } \\ \hline
\end{tabular}
 Description \\
{\footnotesize
Log in to whatever VPNs are necessary to access to the summit notebook
aspect of the RSP

}
\hdashrule[0.5ex]{\textwidth}{1pt}{3mm}
  Expected Result \\
{\footnotesize
VPN connection is active

}
\hdashrule[0.5ex]{\textwidth}{1pt}{3mm}
  Actual Result \\
{\footnotesize

}
\begin{tabular}{p{2cm}p{14cm}}
\toprule
Step 2 & Step Execution Status: \textbf{ Not Executed } \\ \hline
\end{tabular}
 Description \\
{\footnotesize
Log in to the summit notebook aspect: https://summit-lsp.lsst.codes/nb\\
Make sure to choose a recent weekly and a large instance

}
\hdashrule[0.5ex]{\textwidth}{1pt}{3mm}
  Expected Result \\
{\footnotesize
The JupyterLab interface is displayed in the browser

}
\hdashrule[0.5ex]{\textwidth}{1pt}{3mm}
  Actual Result \\
{\footnotesize

}
\begin{tabular}{p{2cm}p{14cm}}
\toprule
Step 3 & Step Execution Status: \textbf{ Not Executed } \\ \hline
\end{tabular}
 Description \\
{\footnotesize
Open a notebook:

\begin{enumerate}
\tightlist
\item
  Navigate to the File-\textgreater{}New-\textgreater{}Notebook
\item
  When prompted, select the LSST kernel
\end{enumerate}

}
\hdashrule[0.5ex]{\textwidth}{1pt}{3mm}
  Expected Result \\
{\footnotesize
An empty notebook running in the LSST kernel

}
\hdashrule[0.5ex]{\textwidth}{1pt}{3mm}
  Actual Result \\
{\footnotesize

}
\begin{tabular}{p{2cm}p{14cm}}
\toprule
Step 4 & Step Execution Status: \textbf{ Not Executed } \\ \hline
\end{tabular}
 Description \\
{\footnotesize
Connect to the summit EFD

}
\hdashrule[0.5ex]{\textwidth}{1pt}{3mm}
  Example Code \\
{\footnotesize
from lsst\_efd\_client import EfdClient\\
efd = EfdClient('summit\_efd')

}
\hdashrule[0.5ex]{\textwidth}{1pt}{3mm}
  Expected Result \\
{\footnotesize
A notebook with an instance of the `EfdClient` configured to talk to the
summit EFD

}
\hdashrule[0.5ex]{\textwidth}{1pt}{3mm}
  Actual Result \\
{\footnotesize

}
\begin{tabular}{p{2cm}p{14cm}}
\toprule
Step 5 & Step Execution Status: \textbf{ Not Executed } \\ \hline
\end{tabular}
 Description \\
{\footnotesize
Choose a topic to query and select a 5 day window of data. The topic and
window are somewhat arbitrary, but it shall be explicit (not relative to
now()) so that it can reproduced. The topic shall be a low cadence topic
with reasonably even sampling. ~I.e. not a command or event topic that
could be very sparse and unevenly sampled. ~The window should be chosen
specifically to be during a period where high cadence topics are not
publishing, so as to measure the latency of the low cadence topics on a
quiet network

}
\hdashrule[0.5ex]{\textwidth}{1pt}{3mm}
  Expected Result \\
{\footnotesize
\begin{itemize}
\tightlist
\item
  A table-like object in memory containing data from the chosen topic
  and time window.
\item
  The window and topic are artifacts to be preserved
\end{itemize}

}
\hdashrule[0.5ex]{\textwidth}{1pt}{3mm}
  Actual Result \\
{\footnotesize

}
\begin{tabular}{p{2cm}p{14cm}}
\toprule
Step 6 & Step Execution Status: \textbf{ Not Executed } \\ \hline
\end{tabular}
 Description \\
{\footnotesize
\begin{enumerate}
\tightlist
\item
  The total latency is the time from the message being published,
  private\_sndStamp, and when it is ingested in the influxDB database.
  Currently the index timestamp is private\_sndStamp There will be an
  additional field added to the measurement to reflect the ingest
  timestamp. This may be technically difficult, in which case a subset
  of topics will have an extra column added by hand for the purpose of
  this analysis. Care shall also be taken to ensure the messages are in
  the same (TAI) time system. In the past some CSCs have been reporting
  TAI and some report UTC. Currently, the difference is 37 seconds.
\item
  Compute the total latency by taking the difference of the two columns,
  arr{[}`timestamp'{]} - arr{[}`private\_sndStamp'{]} (this is where
  correction for TAI/UTC would be included if necessary). The result
  shall be in seconds.
\end{enumerate}

}
\hdashrule[0.5ex]{\textwidth}{1pt}{3mm}
  Expected Result \\
{\footnotesize
An array-like object in memory containing the latency in seconds for
every message in the window

}
\hdashrule[0.5ex]{\textwidth}{1pt}{3mm}
  Actual Result \\
{\footnotesize

}
\begin{tabular}{p{2cm}p{14cm}}
\toprule
Step 7 & Step Execution Status: \textbf{ Not Executed } \\ \hline
\end{tabular}
 Description \\
{\footnotesize
Count the number of entries less than XX second(s) and divide that by
the total number of entries. This value shall be greater than or equal
to XX.

}
\hdashrule[0.5ex]{\textwidth}{1pt}{3mm}
  Expected Result \\
{\footnotesize
\begin{itemize}
\tightlist
\item
  A plot showing a histogram of the latency values indicating the XX\%
  value.
\item
  If the XX\% latency is greater than XX sec, an explanation shall be
  supplied describing why the latency is higher than expected more often
  than expected
\item
  The plot shall also indicate the maximum latency observed
\item
  If the maximum latency is above the nominal maximum (XX sec), an
  explanation shall be provided
\end{itemize}

}
\hdashrule[0.5ex]{\textwidth}{1pt}{3mm}
  Actual Result \\
{\footnotesize

}
\begin{tabular}{p{2cm}p{14cm}}
\toprule
Step 8 & Step Execution Status: \textbf{ Not Executed } \\ \hline
\end{tabular}
 Description \\
{\footnotesize
Document the procedure including latency distributions, time window, and
topics

}
\hdashrule[0.5ex]{\textwidth}{1pt}{3mm}
  Expected Result \\
{\footnotesize
\begin{itemize}
\tightlist
\item
  A document describing the process including the topic chosen and the
  time window.
\item
  The document shall be in the form on a notebook with saved outputs, or
  an instance of an nbreport.
\end{itemize}

}
\hdashrule[0.5ex]{\textwidth}{1pt}{3mm}
  Actual Result \\
{\footnotesize

}

\paragraph{ LVV-T2116 - Verify telemetry is uninterrupted for 5 days and can be analyzed via a
notebook at NCSA }\mbox{}\\

Version \textbf{1}.
Open  \href{https://jira.lsstcorp.org/secure/Tests.jspa#/testCase/LVV-T2116}{\textit{ LVV-T2116 } }
test case in Jira.

Test that telemetry is being recorded without missing messages for 5
days. ~This analysis is to be carried out using NCSA infrastructure.

\textbf{ Preconditions}:\\


Execution status: {\bf Not Executed }

Final comment:\\


Detailed steps results:

\begin{tabular}{p{2cm}p{14cm}}
\toprule
Step 1 & Step Execution Status: \textbf{ Not Executed } \\ \hline
\end{tabular}
 Description \\
{\footnotesize
Log in to whatever VPNs are necessary to access to the NCSA notebook
aspect of the RSP

}
\hdashrule[0.5ex]{\textwidth}{1pt}{3mm}
  Expected Result \\
{\footnotesize
VPN connection is active

}
\hdashrule[0.5ex]{\textwidth}{1pt}{3mm}
  Actual Result \\
{\footnotesize

}
\begin{tabular}{p{2cm}p{14cm}}
\toprule
Step 2 & Step Execution Status: \textbf{ Not Executed } \\ \hline
\end{tabular}
 Description \\
{\footnotesize
Log in to the NCSA notebook aspect:
https://lsst-lsp-stable.ncsa.illinois.edu/nb/\\
Make sure to choose a recent weekly and a large instance

}
\hdashrule[0.5ex]{\textwidth}{1pt}{3mm}
  Expected Result \\
{\footnotesize
The JupyterLab interface is displayed in the browser

}
\hdashrule[0.5ex]{\textwidth}{1pt}{3mm}
  Actual Result \\
{\footnotesize

}
\begin{tabular}{p{2cm}p{14cm}}
\toprule
Step 3 & Step Execution Status: \textbf{ Not Executed } \\ \hline
\end{tabular}
 Description \\
{\footnotesize
Open a notebook:

\begin{enumerate}
\tightlist
\item
  Navigate to the File-\textgreater{}New-\textgreater{}Notebook
\item
  When prompted, select the LSST kernel
\end{enumerate}

}
\hdashrule[0.5ex]{\textwidth}{1pt}{3mm}
  Expected Result \\
{\footnotesize
An empty notebook running in the LSST kernel

}
\hdashrule[0.5ex]{\textwidth}{1pt}{3mm}
  Actual Result \\
{\footnotesize

}
\begin{tabular}{p{2cm}p{14cm}}
\toprule
Step 4 & Step Execution Status: \textbf{ Not Executed } \\ \hline
\end{tabular}
 Description \\
{\footnotesize
Connect to the NCSA EFD\\
Note:The efd identifier is yet to be determined, but shall be ldf\_efd
or similar.

}
\hdashrule[0.5ex]{\textwidth}{1pt}{3mm}
  Example Code \\
{\footnotesize
from lsst\_efd\_client import EfdClient\\
efd = EfdClient('ldf\_efd')

}
\hdashrule[0.5ex]{\textwidth}{1pt}{3mm}
  Expected Result \\
{\footnotesize
A notebook with an instance of the `EfdClient` configured to talk to the
NCSA EFD

}
\hdashrule[0.5ex]{\textwidth}{1pt}{3mm}
  Actual Result \\
{\footnotesize

}
\begin{tabular}{p{2cm}p{14cm}}
\toprule
Step 5 & Step Execution Status: \textbf{ Not Executed } \\ \hline
\end{tabular}
 Description \\
{\footnotesize
Choose a topic to query and select a 5 day window of data. The window is
arbitrary, but must be explicit (not relative to now()) so that it can
be reproduced. Note that command topics shall be avoided for this
purpose since the semantics of private\_seqNum are different in the
context of commands than for other topic types

}
\hdashrule[0.5ex]{\textwidth}{1pt}{3mm}
  Expected Result \\
{\footnotesize
A time window and topic name that will be queried for the sequence
number

}
\hdashrule[0.5ex]{\textwidth}{1pt}{3mm}
  Actual Result \\
{\footnotesize

}
\begin{tabular}{p{2cm}p{14cm}}
\toprule
Step 6 & Step Execution Status: \textbf{ Not Executed } \\ \hline
\end{tabular}
 Description \\
{\footnotesize
Care shall be taken to fix up the id column, private\_seqNum. It must be
monotonically increasing, but gets reset when the CSC is restarted. When
this happens an offset must be applied to the rest of the message ids in
order to produce a monotonically increasing sequence

}
\hdashrule[0.5ex]{\textwidth}{1pt}{3mm}
  Expected Result \\
{\footnotesize
A table-like object in memory with monotonically increasing message
numbers

}
\hdashrule[0.5ex]{\textwidth}{1pt}{3mm}
  Actual Result \\
{\footnotesize

}
\begin{tabular}{p{2cm}p{14cm}}
\toprule
Step 7 & Step Execution Status: \textbf{ Not Executed } \\ \hline
\end{tabular}
 Description \\
{\footnotesize
Compute the difference between message ids via diff =
arr{[}`private\_seqNum'{]}{[}1:{]} - arr{[}`private\_seqNum'{]}{[}:-1{]}

}
\hdashrule[0.5ex]{\textwidth}{1pt}{3mm}
  Expected Result \\
{\footnotesize
A histogram normalized to the total number of samples from 1 to the
maximum number of the array diff.

}
\hdashrule[0.5ex]{\textwidth}{1pt}{3mm}
  Actual Result \\
{\footnotesize

}
\begin{tabular}{p{2cm}p{14cm}}
\toprule
Step 8 & Step Execution Status: \textbf{ Not Executed } \\ \hline
\end{tabular}
 Description \\
{\footnotesize
Confirm that all values are one. Any larger numbers would indicate a gap
in messages

}
\hdashrule[0.5ex]{\textwidth}{1pt}{3mm}
  Expected Result \\
{\footnotesize
The result is that all differences in the histogram should be 1. Any
values larger than 1, which indicates a gap in messages, must be traced
to a problem with the CSC and not an issue with the EFD.

}
\hdashrule[0.5ex]{\textwidth}{1pt}{3mm}
  Actual Result \\
{\footnotesize

}
\begin{tabular}{p{2cm}p{14cm}}
\toprule
Step 9 & Step Execution Status: \textbf{ Not Executed } \\ \hline
\end{tabular}
 Description \\
{\footnotesize
Produce a report outlining this test

}
\hdashrule[0.5ex]{\textwidth}{1pt}{3mm}
  Expected Result \\
{\footnotesize
\begin{itemize}
\tightlist
\item
  A document describing the process including which topics/fields where
  used and what time window was selected.
\item
  The document shall be in the form of a notebook with saved outputs, or
  an instance of an nbreport
\end{itemize}

}
\hdashrule[0.5ex]{\textwidth}{1pt}{3mm}
  Actual Result \\
{\footnotesize

}

\paragraph{ LVV-T2115 - Verify telemetry is uninterrupted for 5 days and can be analyzed via a
notebook at the summit }\mbox{}\\

Version \textbf{1}.
Open  \href{https://jira.lsstcorp.org/secure/Tests.jspa#/testCase/LVV-T2115}{\textit{ LVV-T2115 } }
test case in Jira.

Test that telemetry is being recorded without missing messages for 5
days. ~This analysis is to be carried out using summit infrastructure.

\textbf{ Preconditions}:\\


Execution status: {\bf Not Executed }

Final comment:\\


Detailed steps results:

\begin{tabular}{p{2cm}p{14cm}}
\toprule
Step 1 & Step Execution Status: \textbf{ Not Executed } \\ \hline
\end{tabular}
 Description \\
{\footnotesize
Log in to whatever VPNs are necessary to access to the summit notebook
aspect of the RSP

}
\hdashrule[0.5ex]{\textwidth}{1pt}{3mm}
  Expected Result \\
{\footnotesize
VPN connection is active

}
\hdashrule[0.5ex]{\textwidth}{1pt}{3mm}
  Actual Result \\
{\footnotesize

}
\begin{tabular}{p{2cm}p{14cm}}
\toprule
Step 2 & Step Execution Status: \textbf{ Not Executed } \\ \hline
\end{tabular}
 Description \\
{\footnotesize
Log in to the summit notebook aspect: https://summit-lsp.lsst.codes/nb\\
Make sure to choose a recent weekly and a large instance

}
\hdashrule[0.5ex]{\textwidth}{1pt}{3mm}
  Expected Result \\
{\footnotesize
The JupyterLab interface is displayed in the browser

}
\hdashrule[0.5ex]{\textwidth}{1pt}{3mm}
  Actual Result \\
{\footnotesize

}
\begin{tabular}{p{2cm}p{14cm}}
\toprule
Step 3 & Step Execution Status: \textbf{ Not Executed } \\ \hline
\end{tabular}
 Description \\
{\footnotesize
Open a notebook:

\begin{enumerate}
\tightlist
\item
  Navigate to the File-\textgreater{}New-\textgreater{}Notebook
\item
  When prompted, select the LSST kernel
\end{enumerate}

}
\hdashrule[0.5ex]{\textwidth}{1pt}{3mm}
  Expected Result \\
{\footnotesize
An empty notebook running in the LSST kernel

}
\hdashrule[0.5ex]{\textwidth}{1pt}{3mm}
  Actual Result \\
{\footnotesize

}
\begin{tabular}{p{2cm}p{14cm}}
\toprule
Step 4 & Step Execution Status: \textbf{ Not Executed } \\ \hline
\end{tabular}
 Description \\
{\footnotesize
Connect to the summit EFD

}
\hdashrule[0.5ex]{\textwidth}{1pt}{3mm}
  Example Code \\
{\footnotesize
from lsst\_efd\_client import EfdClient\\
efd = EfdClient('summit\_efd')

}
\hdashrule[0.5ex]{\textwidth}{1pt}{3mm}
  Expected Result \\
{\footnotesize
A notebook with an instance of the `EfdClient` configured to talk to the
summit EFD

}
\hdashrule[0.5ex]{\textwidth}{1pt}{3mm}
  Actual Result \\
{\footnotesize

}
\begin{tabular}{p{2cm}p{14cm}}
\toprule
Step 5 & Step Execution Status: \textbf{ Not Executed } \\ \hline
\end{tabular}
 Description \\
{\footnotesize
Choose a topic to query and select a 5 day window of data. The window is
arbitrary, but must be explicit (not relative to now()) so that it can
be reproduced. Note that command topics shall be avoided for this
purpose since the semantics of private\_seqNum are different in the
context of commands than for other topic types

}
\hdashrule[0.5ex]{\textwidth}{1pt}{3mm}
  Expected Result \\
{\footnotesize
A time window and topic name that will be queried for the sequence
number

}
\hdashrule[0.5ex]{\textwidth}{1pt}{3mm}
  Actual Result \\
{\footnotesize

}
\begin{tabular}{p{2cm}p{14cm}}
\toprule
Step 6 & Step Execution Status: \textbf{ Not Executed } \\ \hline
\end{tabular}
 Description \\
{\footnotesize
Care shall be taken to fix up the id column, private\_seqNum. It must be
monotonically increasing, but gets reset when the CSC is restarted. When
this happens an offset must be applied to the rest of the message ids in
order to produce a monotonically increasing sequence

}
\hdashrule[0.5ex]{\textwidth}{1pt}{3mm}
  Expected Result \\
{\footnotesize
A table-like object in memory with monotonically increasing message
numbers

}
\hdashrule[0.5ex]{\textwidth}{1pt}{3mm}
  Actual Result \\
{\footnotesize

}
\begin{tabular}{p{2cm}p{14cm}}
\toprule
Step 7 & Step Execution Status: \textbf{ Not Executed } \\ \hline
\end{tabular}
 Description \\
{\footnotesize
Compute the difference between message ids via diff =
arr{[}`private\_seqNum'{]}{[}1:{]} - arr{[}`private\_seqNum'{]}{[}:-1{]}

}
\hdashrule[0.5ex]{\textwidth}{1pt}{3mm}
  Expected Result \\
{\footnotesize
A histogram normalized to the total number of samples from 1 to the
maximum number of the array diff.

}
\hdashrule[0.5ex]{\textwidth}{1pt}{3mm}
  Actual Result \\
{\footnotesize

}
\begin{tabular}{p{2cm}p{14cm}}
\toprule
Step 8 & Step Execution Status: \textbf{ Not Executed } \\ \hline
\end{tabular}
 Description \\
{\footnotesize
Confirm that all values are one. Any larger numbers would indicate a gap
in messages

}
\hdashrule[0.5ex]{\textwidth}{1pt}{3mm}
  Expected Result \\
{\footnotesize
The result is that all differences in the histogram should be 1. Any
values larger than 1, which indicates a gap in messages, must be traced
to a problem with the CSC and not an issue with the EFD.

}
\hdashrule[0.5ex]{\textwidth}{1pt}{3mm}
  Actual Result \\
{\footnotesize

}
\begin{tabular}{p{2cm}p{14cm}}
\toprule
Step 9 & Step Execution Status: \textbf{ Not Executed } \\ \hline
\end{tabular}
 Description \\
{\footnotesize
Produce a report outlining this test

}
\hdashrule[0.5ex]{\textwidth}{1pt}{3mm}
  Expected Result \\
{\footnotesize
\begin{itemize}
\tightlist
\item
  A document describing the process including which topics/fields where
  used and what time window was selected.
\item
  The document shall be in the form of a notebook with saved outputs, or
  an instance of an nbreport
\end{itemize}

}
\hdashrule[0.5ex]{\textwidth}{1pt}{3mm}
  Actual Result \\
{\footnotesize

}




\newpage
\appendix
% Make sure lsst-texmf/bin/generateAcronyms.py is in your path
\section{Acronyms used in this document}\label{sec:acronyms}
\input{acronyms.tex}

\newpage

% generated from JIRA project LVV
% using template at /Users/krughoff/lsst_stack/conda/miniconda3-py38_4.9.2/envs/docsteady-env/lib/python3.7/site-packages/docsteady/templates/tpr-appendix.latex.jinja2.
% using docsteady version 2.2.3
% Please do not edit -- update information in Jira instead
\section{Traceability}

\begin{longtable}{p{3cm}p{3cm}L{9cm}}
\hline
\textbf{Test Case} & \textbf{VE Key} & \textbf{VE Summary} \\ \hline
\href{https://jira.lsstcorp.org/secure/Tests.jspa#/testCase/LVV-T2111}{LVV-T2111} &
 & \\ \hline
\href{https://jira.lsstcorp.org/secure/Tests.jspa#/testCase/LVV-T2112}{LVV-T2112} &
 & \\ \hline
\href{https://jira.lsstcorp.org/secure/Tests.jspa#/testCase/LVV-T2115}{LVV-T2115} &
 & \\ \hline
\href{https://jira.lsstcorp.org/secure/Tests.jspa#/testCase/LVV-T2116}{LVV-T2116} &
 & \\ \hline
\href{https://jira.lsstcorp.org/secure/Tests.jspa#/testCase/LVV-T2117}{LVV-T2117} &
 & \\ \hline
\end{longtable}


\end{document}
